\documentclass[11pt]{article}

\usepackage{hyperref}
\usepackage{xcolor}
\usepackage{calc}
\usepackage{graphicx}
\usepackage{tikz}
\usepackage[fontset=none]{ctex}
\usepackage{titlesec}
\usepackage{enumitem}
\usepackage{fancybox}

% 使用 Google Fonts 的 Material Symbols 图标库
\usepackage{material-symbols}

\hypersetup{hidelinks}

%%%%%%%%%%%%%%%%%%%%
% 全局样式设置
%%%%%%%%%%%%%%%%%%%%

\setlength{\parindent}{0pt}					% 取消全局段落缩进
\pagenumbering{gobble}						% 取消页码显示
\setlist[itemize]{nosep                     % 取消 itemize 的默认间距
    , before={\vspace*{-\parskip}}          % 取消 itemize 和后续段落之间的空白
    , leftmargin=*}		                    % 取消 itemize 的左边距
\setlist[enumerate]{leftmargin=*}	        % 取消 enumerate 的左边距
\renewcommand{\arraystretch}{1.2}           % 设置表格行间距
\linespread{1.25}                           % 设置正文行间距

% 苏州大学色彩定义
\definecolor{suda_red}{RGB}{175, 37, 28}    % 苏大红
\definecolor{primary_color}{named}{suda_red}
\definecolor{secondary_color}{named}{suda_red}

% 章节标题样式设置
\titleformat{\section}
  {\LARGE\bfseries\raggedright} 		      % 标题字体:大号、加粗、左对齐
  {}{0em}                      			  % 无序号
  {}                           			  
  [{\color{secondary_color}\titlerule}]     % 标题下方装饰线
\titlespacing*{\section}{0cm}{*1.2}{*1.2}	% 标题边距:左、上、下

\usepackage[
	a4paper,
	left=1.2cm,
	right=1.2cm,
	top=1.5cm,
	bottom=1cm,
	nohead
]{geometry}                                 % 页面边距设置

% 字体配置 (需确保 fonts 目录下存在对应文件)
\setmainfont[
    Path=fonts/,
    Extension=.otf,
    BoldFont=*-Bold,
]{NotoSerifSC}

\setCJKmainfont[
    Path=fonts/,
    Extension=.otf,
    BoldFont=*-Bold,
]{NotoSerifSC}

\newlength{\iconwidth}
\setlength{\iconwidth}{1.5em}                   % 章节图标占用的宽度

%%%%%%%%%%%%%%%%%%%%
% 个人信息宏定义
%%%%%%%%%%%%%%%%%%%%

% 所在学院/单位
\newcommand{\school}{计算机科学与技术学院 | School of Computer Science and Technology} 

% 联系方式页脚内容
\newcommand{\contact}{
    \footnotesize
    \textcolor{white}{
        % 邮箱
        \href{mailto:yourname@suda.edu.cn}{{\normalsize\mSymbol{mail}}\quad yourname@suda.edu.cn}
        \hspace{4em}
        % 手机号
        \href{tel:138-0000-0000}{{\normalsize\mSymbol{call}}\quad 138-0000-0000}
        % GitHub/个人主页
        \hspace{4em}
        \href{https://github.com/yourusername}{{\normalsize\mSymbol{package-2}}\quad GitHub: yourusername}
    }
}

\begin{document}

    %%%%%%%%%%%%%%%%%%%%
    % 视觉元素 (页眉、页脚、水印)
    %%%%%%%%%%%%%%%%%%%%

    % 页眉:苏大红背景条 + 校徽 + 学院名
    \begin{tikzpicture}[remember picture, overlay]
        \fill[suda_red] (current page.north west) rectangle ([yshift=-2.5cm]current page.north east);
        
        % 校徽:圆形白底包裹
        \node[anchor=west, fill=white, inner sep=0pt, circle](school_logo) at ([shift={(0.8cm,-1.25cm)}]current page.north west){
            \includegraphics[height=1.5cm]{figs/suda_logo.pdf}
        };
        
        % 学院名称显示
        \node[anchor=east](school_name) at ([shift={(-0.8cm,-1.25cm)}]current page.north east){
            \textcolor{white}{\textbf{\school}}
        };
    \end{tikzpicture}
    \vspace{0.5em}

    % 页脚:联系方式背景条
    \begin{tikzpicture}[remember picture, overlay]
        \fill[suda_red] (current page.south west) rectangle ([yshift=1.2cm]current page.south east);
        \node[anchor=center] at ([yshift=0.6cm]current page.south){\contact};
    \end{tikzpicture}

    % 背景水印:苏州大学校徽 (低透明度)
    \begin{tikzpicture}[remember picture, overlay]
        \node[opacity=0.06] at(current page.center){
            \includegraphics[width=0.7\paperwidth, keepaspectratio]{figs/suda_logo.pdf}
        };
    \end{tikzpicture}

    %%%%%%%%%%%%%%%%%%%%
    % 简历正文
    %%%%%%%%%%%%%%%%%%%%

    \vspace{2.5em} % 避开页眉装饰条

    \begin{minipage}[t]{0.78\textwidth}
        % 个人基本信息
        \begin{minipage}[t]{\textwidth}
        \section[个人信息]{\makebox[\iconwidth][c]{\color{primary_color}{\Huge\mSymbol{badge}}}\quad 个人信息}
        \begin{minipage}[t]{0.5\textwidth}
            \textbf{姓\qquad 名}:苏小大
            
            \vspace{0.5em}
            \textbf{出生年月}:1900年5月
        \end{minipage}
        \begin{minipage}[t]{0.35\textwidth}
            \textbf{性\qquad 别}:男
            
            \vspace{0.5em}
            \textbf{政治面貌}:中共党员
        \end{minipage}
        \vspace{1.2em}
        \end{minipage}

        % 教育背景
        \begin{minipage}[t]{\textwidth}
        \section[教育背景]{\makebox[\iconwidth][c]{\color{primary_color}{\Huge\mSymbol{school}}}\quad 教育背景}
        
        {\large \textbf{苏州大学}},本科 \hfill 2018年9月--2022年6月
        \begin{itemize}
            \item 计算机科学与技术学院,计算机科学与技术专业
            \item \textbf{主修课程}:数据结构、操作系统、计算机网络、数据库系统等。
            \item \textbf{GPA}:3.9 / 4.0(排名:5 / 150)
        \end{itemize}
        
        \vspace{0.5em}
        {\large \textbf{苏州大学}},硕士 \hfill 2022年9月--至今
        \begin{itemize}
            \item 计算机科学与技术学院,软件工程专业,导师:某某教授
            \item \textbf{研究方向}:人工智能、深度学习、自然语言处理等。
        \end{itemize}
        
        \vspace{1.2em}
        \end{minipage}
    \end{minipage}
    \hfill
    % 个人照片展示
    \begin{minipage}[t]{0.2\textwidth}
        \vspace{2em}
        \setlength{\fboxsep}{0pt}
        \doublebox{\includegraphics[width=\linewidth]{images/avatar.png}}
    \end{minipage}

    \begin{minipage}[t]{\textwidth}
    % 科研成果展示
    \section[科研成果]{\makebox[\iconwidth][c]{\color{primary_color}{\Huge\mSymbol{contract-edit}}}\quad 科研成果}

    Title of Your Research Paper Published in Top Venue.
    \begin{itemize}
        \item \textbf{Xiao-Da Su}, Professor Someone. \hfill 发表于 \textbf{NeurIPS 2024}
        \item 简要描述该研究的主要贡献,例如提出了一种新型的神经网络架构,在多项基准测试中达到了 SOTA 性能。
    \end{itemize}

    \vspace{1.2em}
    \end{minipage}

    \begin{minipage}[t]{\textwidth}
    % 竞赛获奖情况
    \section[竞赛获奖]{\makebox[\iconwidth][c]{\color{primary_color}{\Huge\mSymbol{emoji-events}}}\quad 竞赛获奖}

    \begin{itemize}
        \item \textbf{第十八届“挑战杯”全国大学生课外学术科技作品竞赛} \hfill \textbf{全国特等奖} (2023年)
        \item \textbf{美国大学生数学建模竞赛 (MCM/ICM)} \hfill \textbf{Finalist (特等奖提名)} (2022年)
        \item \textbf{中国大学生计算机设计大赛} \hfill \textbf{一等奖} (2021年)
    \end{itemize}

    \vspace{1.2em}
    \end{minipage}

    \begin{minipage}[t]{\textwidth}
    % 项目经历/社会实践
    \section[项目经历]{\makebox[\iconwidth][c]{\color{primary_color}{\Huge\mSymbol{co-present}}}\quad 项目经历}
    
    {\large \textbf{智能苏州大学校园导航系统}} \hfill 2023年3月--2023年12月
    \begin{itemize}
        \item \textbf{项目负责人} \hfill 苏州大学大学生创新创业项目
        \item 基于跨平台框架开发,实现了校园室内外无缝导航,服务学生超过 5000 人次。
    \end{itemize}

    \vspace{1.2em}
    \end{minipage}
    
    % 底部两列布局:技能与兴趣
    \begin{minipage}[t]{0.6\textwidth}
        \section[技能特长]{\makebox[\iconwidth][c]{\color{primary_color}{\Huge\mSymbol{handyman}}}\quad 技能特长}
        \begin{itemize}
        \setlength{\itemsep}{0.5em}
            \item 熟练掌握 C/C++、Python、Java 等开发语言。
            \item 深入理解 PyTorch、TensorFlow 等主流深度学习框架。
            \item 具备良好的英语读写能力(CET-6 600+)。
        \end{itemize}
    \end{minipage}
    \hfill
    \begin{minipage}[t]{0.35\textwidth}
        \section[兴趣爱好]{\makebox[\iconwidth][c]{\color{primary_color}{\Huge\mSymbol{star}}}\quad 兴趣爱好}
        \begin{itemize}
        \setlength{\itemsep}{0.5em}
            \item 摄影与后期处理
            \item 马拉松长跑
            \item 开源社区贡献
        \end{itemize}
    \end{minipage}
    
\end{document}
